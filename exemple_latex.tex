\documentclass[10pt]{beamer}
\usepackage[utf8]{inputenc}
\usepackage[T1]{fontenc}
\usepackage{listings}

\title{Un cours intéressant}
\subtitle{et didactique !}
\date[2019]{Novembre 2019}
\author[Euclid]{Euclid of Alexandria \texttt{euclid@alexandria.edu}}
\institute{Département mathématique d'Alexandrie}
\usetheme{sorbonne}

\begin{document}

\begin{frame}[plain]
\titlepage
\end{frame}

\begin{frame}
        \frametitle{Sommaire}
        \tableofcontents[hideallsubsections]
\end{frame}

\section{Introduction avec un titre un peu long mais qui ne casse pourtant pas trois pattes à un canard}

\subsection{Commençons par le commencement}

\begin{frame} 
\frametitle{There Is No Largest Prime Number} 
\framesubtitle{The proof uses \textit{reductio ad absurdum}.} 
\begin{theorem}
	There is no largest prime number. \end{theorem} 
\begin{enumerate} 
	\item<1-| alert@1> Suppose $p$ were the largest prime number. 
	\item<2-> Let $q$ be the product of the first $p$ numbers. 
	\item<3-> Then $q+1$ is not divisible by any of them. 
	\item<1-> But $q + 1$ is greater than $1$, thus divisible by some prime
	number not in the first $p$ numbers.
\end{enumerate}
\end{frame}

\subsection{Finissons !}

\begin{frame}{Blocks}

\begin{block}{Bloc standard}
  $$1 + e^{i \pi} = 0.$$
\end{block}

\begin{alertblock}{Alerte !}
$$P(x) = \frac{1}{{\sigma \sqrt {2\pi } }}e^{{{ - \left( {x - \mu } \right)^2 } \mathord{\left/ {\vphantom {{ - \left( {x - \mu } \right)^2 } {2\sigma ^2 }}} \right. \kern-\nulldelimiterspace} {2\sigma ^2 }}}$$
\end{alertblock}
\begin{exampleblock}{Ca par exemple !}
  $$E = m C^2$$
\end{exampleblock}

\end{frame}

\begin{frame}[fragile]{Code}
\begin{lstlisting}

// Gestion du contexte ete2013
// Utilisation d'un nouveau template
if (%variables['ctpage'] == "ete2013") {
    variables['template_files']=array('page-ete');
}
\end{lstlisting}
\end{frame}
\end{document}

